"Post exploitation" refers to the stage where an attacker has gained access to a vulnerable machine, and seeks to maintain
his privileges for further utilization. For anti-malware softwares, the primary objective is to prevent malwares from being
executed, so attackers could be stopped from maintaining access. In recent years, the detection techniques against malwares on the Windows platform have matured.
Linux, on the other hand, due to its open-source and non-commercial property, the research of malware detection for the
platform is developing much slower. However, Linux is the most widely used platform for servers and data centers, so the
study on post exploitation is not only important, but also urgent.

This thesis presents an effective way of hiding malwares from anti-malware softwares after researching into the common
malware detection techniques used on Linux platforms. Most malwares in the wild works by executing a seperate file
containing the actual attack payload in order to maintain control of the vulnerable machine, yet Linux provides only a
small set of syscalls for processes to achieve that. As a result, anti-malware softwares could easily detect them by closely
monitoring the activities of the syscalls like open and execve. This thesis, however, designed a malware that loads
the ELF which contains the actual payload into it's own virtual memory space, and prevented calling the sensitive syscalls
by imitating their behaviors. This made it harder for anti-malware softwares to detect such malwares.

The method presented in this thesis helped a malware demo to achieve bypassing the detection conducted by over sixty
malware detection engines including Kaspersky, McAfee and Tencent, while it successfully loads and executes the payload
sent to it. Finally, this thesis conducted an analysis on the method, and presented some ideas on how a similar threat could be
mitigated or eliminated. This thesis also concluded some of the obstacles researchers may encounter while strengthening
Linux security.
