\chapter{绪论}

\section{恶意程序隐蔽技术研究背景}
% 为什么要研究隐蔽和反隐蔽
随着计算机网络系统的发展,计算机网络攻击技术也由简单的单一类型攻击,发展到复杂化和理论化的网络渗透(Network Penetration)\cite{Mirjalili:2014}
技术。网络渗透技术是一种综合的高级攻击技术,具有分阶段、分层次和攻击手段多样性的特点。而后渗透(Post-Peneration)\cite{ENGEBRETSON2013167}阶段是指渗透阶段
完成、攻击者已经进入到目标操作系统后,攻击者需要稳定地维持自己权限的阶段。该阶段的攻击者往往需要向目标系统注入如木马等恶意程序,以达到向远程
服务器注入恶意程序或命令等目标。所以后渗透阶段的攻击和防御,往往体现在恶意程序的隐蔽和反隐蔽的研究。

% 为什么是Linux:因为重要
Linux系统以其开放、透明、易用的优势占有了服务端计算机系统绝大多数的市场份额\cite{LinuxShare},并占据了绝对的主导地位。可以肯定,Linux
在当代互联网中具有极为重要的意义。但在Linux操作系统主导服务端操作系统市场的同时,也是除桌面系统外最主要的攻击目标之一。
这又使得针对Linux环境的恶意程序隐蔽和反隐蔽研究具有很高的理论价值。

% 为什么是Linux:因为简单并且可以推导至其他系统
相比于Windows操作系统,Linux系统提供了完善的源码以及文档以供研究,因此在Linux系统上进行恶意程序的隐蔽和反隐蔽研究更为简单,同时Linux
系统和其他操作系统又有很多相似之处,这使得对Linux系统的研究又不失适用性。

% 这个背景和我们做的事有什么关系
据此,本文以Linux系统上常见的恶意程序的常见的隐蔽和反隐蔽手段为研究方向,并通过劫持和伪造文件相关结构体并避免使用敏感系统调用,设计了Linux
平台上恶意程序装载器,以规避大部分恶意程序反隐蔽程序的反隐蔽。从而设计了一种恶意程序的隐蔽方法。同时针对该方法提出了一些反制思路,并对Linux平台恶意程序反隐蔽技术发展的前景做出了展望。

\section{恶意程序反隐蔽技术研究现状和发展趋势}

恶意程序隐蔽方法的研究往往针对现有的恶意程序反隐蔽技术进行。所以对恶意程序隐蔽方法的研究,也往往与恶意程序反隐蔽技术相关。而恶意程序的
反隐蔽技术随着互联网的发展也在不断变化。互联网发展早期,恶意程序反隐蔽技术以传统的基于行为和特征的恶意程序反隐蔽技术为主;但随着人工
智能、机器学习和数据挖掘等技术的发展,基于这些领域的恶意程序反隐蔽技术也随之兴起。下文将详细阐述国内外相关领域的研究现状和发
展趋势。

\subsection{传统恶意程序反隐蔽技术}

传统的恶意程序反隐蔽技术主要可以分为两类:基于特征和基于行为的恶意程序反隐蔽技术。互联网发展早期,恶意程序反隐蔽技术以传统的特征检测为主
\cite{汪嘉来:977}。特征检测方法效率高,准确度高,且在工程上具有较好的扩展性。但它们需要人工制定许多检测规则,并且基于特征检测的
反隐蔽方法无法检测出未知的恶意程序,当恶意程序发生了变化,特征检测方法除了增加对应的检测规则外便无能为力\cite{Ye:2017}。

针对基于特征检测的反隐蔽技术的不足,基于行为反隐蔽的技术应运而生。基于行为的恶意程序反隐蔽技术又可以分为两类,基于异常(Anomaly-based)和基于规则
(Specification-based)的反隐蔽技术,它们分别通过构建关于恶意程序的共通行为和关于正常程序的共通行为的知识来检测恶意程序,比如Mansour等人早在
2013年就提出了通过分析程序的行为序列\cite{AHMADI201311},来判断程序是否为恶意程序。2020年,Vurdelja等人总结了Linux操作系统中跟踪程序行为常用
的方法与工具\cite{vurdelja2020detection}。很多研究者也将程序完整性运行时检测\cite{LinuxTrustVeri}\cite{Duong2021DetectingMB}
与调试技术应用到了对恶意程序的反隐蔽中。可以预见,随着硬件水平的不断提高,程序分析技术在恶意程序反隐蔽中的高强度应用也将成为可能。

\subsection{新型恶意程序反隐蔽技术}

现有的大多数新型反隐蔽手段,包括使用人工智能、机器学习数据挖掘等分析手段,都需要以目标程序的行为或者特征为依据进行判断\cite{MalDetectReview}。
这些新技术的使用给传统反隐蔽技术注入了新的活力,理论上可以减少人工添加恶意程序文件特征或恶意程序行为特征规则的工作。

\subsection{不同平台下恶意程序隐蔽和反隐蔽技术的差异}

Linux下的恶意程序防护目前有下面三个特征:
\begin{enumerate}
\item 市面上Linux恶意程序隐蔽技术没有Windows下成熟:其主要原因为Windows常被当作主要的攻击目标,且Windows为闭源程序,内核较为复杂、
        攻击面丰富,而Linux内核源码向所有人公开,易于审计与发现安全漏洞。另一方面,由于Windows作为人们主要的桌面工作环境,其利用价
        值往往较功能单一的服务器更高;
\item Linux环境复杂多样,不同发行版间差异较大:这导致了防护程序需要处理不同发行版细节上的差异,提高了通用防护的成本;
\item Linux内核在版本迭代的过程中增添了很多功能各异的系统调用:这加大了恶意程序的攻击面,使得恶意程序能通过利用新式攻击途径绕过传统
        的基于规则的防护措施。但同时由于特征明显,启发式的防护措施能轻易地发现并阻止这一类恶意行为。
\end{enumerate}

但现有的以隐蔽地加载程序为主要目标的装载器主要存在于Windows平台,如能够从内存中装载PE文件的MemoryModule\cite{MemModule}项目获得了较大关注。
Linux下装载器亦有很多不同的实现,但大部分的实现的关注点主要围绕于在有特殊需求(如低功耗)的操作系统或标准库不完整的设备上能否稳定运行,而非为了隐
蔽所加载的程序,如elfload\cite{elfload}、嵌入式程序加载技术研究\cite{EmbedLoad},但其成果对恶意程序装载恶意代码亦有一定的参考价值。同时,
官方文档\cite{ELFFormat}与早期对ELF设计的研究与讨论\cite{StudyElfReloc}也有很大的参考意义。如果恶意程序能隐蔽地装载通用的、由使用广泛的
编译器编译出的ELF文件,将会大大减少恶意程序中包含的特征,提高其隐蔽效果。

\section{研究切入点与意义}

\begin{enumerate}
\item 通过研究Linux系统下恶意程序的隐蔽手段,结合现有的恶意程序样本所采用的手段,能够分析恶意程序隐蔽技术的共性与特点,总结出
    进一步加强Linux系统保护的思路与措施,生产出更为有效的保护程序。本研究还期望从对Linux系统代码的研究中挖掘出更多的成果与结论;
\item 在渗透测试工作中,目标系统中往往存在大量因利用条件苛刻而无法成功利用的潜在漏洞。这一类漏洞虽然客观存在,但由于在渗透测试中
    无法成功利用而无法引起足够的重视,进而导致漏洞没有道到重视而及时修补,使得Linux系统长期且持续地暴露在安全事件中而无法得到有效的保护。本研究期望
    减少基于规则的保护对渗透测试与研究人员带来的额外攻击成本,使测试与研究人员能够在有限的时间内专注于漏洞本身,提高测试工作的效率,
    使潜在的安全漏洞得到足够的重视并得以修补,从而更全面地加强系统的安全性。
\end{enumerate}
