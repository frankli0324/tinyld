后渗透阶段指攻击者维持非法权限并进行攻击的阶段,该阶段系统防御的主要目的是检测攻击者在目标计算机上运行的恶意程序。而Linux平台
的非商业性导致该平台上的恶意程序检测技术相对不足,但Linux平台又是最为广泛使用的计算机网络服务器平台,所以对Linux平台上后渗透阶
段的研究,也就是该平台上恶意程序的隐蔽和检测方法具有重要意义。

本文通过研究后发现:由于大多数恶意程序进行隐蔽或者攻击行为最终必然要使用到open和execve等系统调用,所以Linux平台上恶意软件的检测手段
往往通过扫描程序代码特征或者监控敏感系统调用来实现。根据以上发现,本文设计了一种能够规避绝大多数Linux平台上恶意软件检测手段的方法:
通过设计一种恶意程序装载器:该装载器模仿部分敏感系统调用的行为,将从网络下载的恶意程序直接装载到自身的虚拟内存空间中,但不调用任何敏感
系统调用,从而在规避上述检测手段的同时达到隐蔽恶意程序的效果。

本文利用提出的恶意程序隐蔽方法编写了验证程序。该程序通过了卡巴斯基、McAfee、Tencent等61个恶意程序分析引擎的检测,证明了Linux平台上恶意程序反
隐蔽手段存在明显缺陷。最后本文分析了该方法的不足并提出了一些针对该方法的反制思路,总结了在Linux平台恶意程序检测技术发展的途中可能遇到的问题与阻碍。

% 本文设计方法能够实现静态恶意程序的有效装载,但尚未实现动态链接程序的加载,有待今后进一步改进和开发。
